\documentclass[12pt,aspectratio=169]{beamer}

\usetheme[progressbar=frametitle, numbering=fraction]{metropolis}
\usepackage{appendixnumberbeamer}
\usepackage{gensymb}
\usepackage{booktabs}
\usepackage[scale=2]{ccicons}

\usepackage{pgfplots}
\usepgfplotslibrary{dateplot}
\usepackage[english]{babel}

\usepackage{xspace}
\newcommand{\themename}{\textbf{\textsc{metropolis}}\xspace}

\usepackage{amsmath}

% Chinese Fonts (Fontset: fandol,ubuntu)
\usepackage[fontset=windows]{ctex}

% Math Fonts
\usefonttheme{professionalfonts}
\usepackage{mathspec}
\setsansfont[BoldFont={Fira Sans},
  Numbers={OldStyle}]{Fira Sans Light}
\setmathsfont(Digits)[Numbers={Lining, Proportional}]{Fira Sans Light}

% Change Color of the theme
\usepackage{xcolor}
\definecolor{DarkGrey}{HTML}{353535}
\definecolor{ECNURed}{RGB}{164,31,53}
\definecolor{ECNUBrown}{RGB}{134,117,77}
\definecolor{BackGround}{RGB}{250,250,250}
\definecolor{MyBlue}{RGB}{0,161,233}
\definecolor{MyRed}{RGB}{228,0,127}
\setbeamercolor{normal text}{ fg= DarkGrey  }
\setbeamercolor{alerted text}{ fg= ECNURed  }
\setbeamercolor{example text}{ fg= ECNUBrown  }

% Bolder Fonts for presenting in a large room 
\setsansfont[BoldFont={Fira Sans SemiBold}]{Fira Sans}
\metroset{block=fill}

\usepackage{listings,xcolor}
\usepackage{tikz}
\usepackage{pgfmath}
\usepackage{animate,media9,graphicx}
\usepackage{calligra}
\usepackage{array}
\renewcommand{\arraystretch}{2}  % 增加行高
\setlength{\tabcolsep}{10pt}     % 增加列间距

\lstset{
  language         = c++,
  numbers          = left,
  numberstyle      = \tiny,
  breaklines       = true,
  captionpos       = b,
  tabsize          = 4,
  frame            = shadowbox,
  columns          = fullflexible,
  commentstyle     = \color[RGB]{0,128,0},
  keywordstyle     = \color[RGB]{0,0,255},
  basicstyle       = \tiny\ttfamily,
  stringstyle      = \color[RGB]{148,0,209}\ttfamily,
  rulesepcolor     = \color{red!20!green!20!blue!20},
  showstringspaces = false,
}

\title{NeuralODE with Knowledge}
\subtitle{}
\author{Yifei Ding}
\date{\today}
% \institute{演讲者描述}
% \titlegraphic{\hfill\includegraphics[height=1.5cm]{ECNUlogo.png}}

\begin{document}

\maketitle
\footnotesize
% \begin{frame}{Contents}
%   \setbeamertemplate{section in toc}[sections numbered]
%   \tableofcontents%[hideallsubsections]
% \end{frame}

\begin{frame}{NeuralODE with Knowledge}

  \[
    \frac{\mathrm{d}^2P(t)}{\mathrm{d}t^2}+2\zeta\omega_n
    \frac{\mathrm{d}P(t)}{\mathrm{d}t}+\omega_n^2P(t)=K\omega_n^2I(t),
  \]

  let $\displaystyle v(t)=\frac{\mathrm{d}P(t)}{\mathrm{d}t}$, then

  \[
    \begin{cases}
      \displaystyle\frac{\mathrm{d}P(t)}{\mathrm{d}t}=v(t),\\
      \displaystyle\frac{\mathrm{d}v(t)}{\mathrm{d}t}+2\zeta\omega_n v(t)+\omega_n^2 P(t)=K\omega_n^2 I(t),
    \end{cases}
  \]

  therefore 
  \[
    \frac{\mathrm{d}}{\mathrm{d}t}\left[\begin{matrix}P(t)\\ v(t)\end{matrix}\right]
      = \left[\begin{matrix}0 & 1 \\ -\omega_n^2 & -2\zeta\omega_n \end{matrix}\right] \left[\begin{matrix}P(t)\\ v(t)\end{matrix}\right]
      + \left[\begin{matrix}0 \\ K\omega_n^2\end{matrix}\right] I(t).
  \]

\end{frame}

\begin{frame}{Acknowledgement}
  \begin{center}
    \textcolor{gray}{\Huge{\centerline{\calligra{Thank you!}}}}
  \end{center}
\end{frame}

\end{document}
