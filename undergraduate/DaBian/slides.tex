\documentclass[12pt,aspectratio=169]{beamer}

\usetheme[progressbar=frametitle, numbering=fraction]{metropolis}
\usepackage{appendixnumberbeamer}
\usepackage{gensymb}
\usepackage{booktabs}
\usepackage[scale=2]{ccicons}

\usepackage{pgfplots}
\usepgfplotslibrary{dateplot}
% \usepackage[english]{babel}
\usepackage{babel}

\usepackage{xspace}
\newcommand{\themename}{\textbf{\textsc{metropolis}}\xspace}

% Chinese Fonts (Fontset: fandol,ubuntu)
\usepackage[fontset=fandol]{ctex}

% Math Fonts
\usefonttheme{professionalfonts} 
\usepackage{mathspec}
\setsansfont[BoldFont={Fira Sans},
Numbers={OldStyle}]{Fira Sans Light}
\setmathsfont(Digits)[Numbers={Lining, Proportional}]{Fira Sans Light}

% Change Color of the theme
\usepackage{xcolor}
\definecolor{DarkGrey}{HTML}{353535}
\definecolor{ECNURed}{RGB}{164,31,53}
\definecolor{ECNUBrown}{RGB}{134,117,77}
\definecolor{BackGround}{RGB}{250,250,250}
\definecolor{MyBlue}{RGB}{0,161,233}
\definecolor{MyRed}{RGB}{228,0,127}
\setbeamercolor{normal text}{ fg= DarkGrey  }
\setbeamercolor{alerted text}{ fg= ECNURed  }
\setbeamercolor{example text}{ fg= ECNUBrown  }

% Bolder Fonts for presenting in a large room 
\setsansfont[BoldFont={Fira Sans SemiBold}]{Fira Sans}
\metroset{block=fill}

\usepackage{listings,xcolor}
\usepackage{tikz}
\usepackage{pgfmath}
\usepackage{animate,media9,graphicx}
\usepackage{calligra}
\usepackage{array}
\renewcommand{\arraystretch}{2}  % 增加行高
\setlength{\tabcolsep}{10pt}       % 增加列间距

\usepackage[backend=biber,style=gb7714-2015,sorting=none]{biblatex}
\setbeamerfont{bibliography item}{size=\tiny}
\setbeamerfont{bibliography entry author}{size=\tiny}
\setbeamerfont{bibliography entry title}{size=\tiny}
\setbeamerfont{bibliography entry location}{size=\tiny}
\setbeamerfont{bibliography entry note}{size=\tiny}
\renewcommand{\bibfont}{\tiny}
\addbibresource{slides.bib}

\lstset{
	language         = c++,
	numbers          = left,
	numberstyle      = \tiny,
	breaklines       = true,
	captionpos       = b,
	tabsize          = 4,
	frame            = shadowbox,
	columns          = fullflexible,
	commentstyle     = \color[RGB]{0,128,0},
	keywordstyle     = \color[RGB]{0,0,255},
	basicstyle       = \tiny\ttfamily,
	stringstyle      = \color[RGB]{148,0,209}\ttfamily,
	rulesepcolor     = \color{red!20!green!20!blue!20},
	showstringspaces = false,
}

\title{基于蒙特卡洛树搜索的等面积划分球面中周长极小值探索}
% \subtitle{}
\date{\today}
\author{汇报人:丁逸飞\ \ \ \ \ \ \ \ \ \ 指导老师:王二小}
% \institute{}
% \titlegraphic{\hfill\includegraphics[height=1.5cm]{ECNUlogo.png}}

\begin{document}

\maketitle
\footnotesize
\begin{frame}{Contents}
  \setbeamertemplate{section in toc}[sections numbered]
  \tableofcontents%[hideallsubsections]
\end{frame}

\section{选题背景与意义}

\begin{frame}{选题背景与意义}

  \begin{block}{蜂巢问题}

    构造一个平面的等面积划分,使得这个划分的总周长尽可能的小. 

  \end{block}

  1999 年,Hales\cite{hales2001honeycomb}证明了最优解为平面的正六边形密铺,同时将其平面上的理论应用到了球
  面之上,证明了对于 $n = 12$ 的球面蜂巢问题,正五边形分割的总周长是最小的. 此后,人
  们相继证明了 $n = 2, 3, 4, 6$ 的几种情况,其他情况都还等待着我们去发现. 

  通过研究球面蜂巢问题,我们既可以深入探索几何排列问题,寻找更一般的排列规律和方
  法,也可以深入探索空间优化理论和方法,对于工程、建筑和计算机图形学等领域的空间布
  局和规划具有非常重要的理论意义. 

\end{frame}

\section{文献综述}

\begin{frame}{经典蜂巢问题研究进展}

  蜂巢问题寻求将平面划分为单位区域的最小周长方法,经过长期推测,蜂巢定理指出,正六
  边形提供了这种最小周长的划分,并由 Hales 于 1999 年证明\cite{hales2001honeycomb}. 对于球体的类似猜测是划
  分为 $n$ 个全等的正 $m$ 边形可以使划分为 $n$ 个相等区域的周长最小化,事实上,只存在五种
  这样的分区. 

  后来,Hales 推广了他的平面方法来证明 $n = 12$ 的情况,即 $12$ 个正五边形的十二面体排列
  为周长最小的情况. 1994 年,Masters\cite{masters1996perimeter}使用肥皂泡理论工具和计算机论证证明了球体上的
  双泡定理,等效解决了 $n = 3$ 时的球面蜂巢问题. 2007 年,Conor Quinn\cite{quinn2007least}为Masters 的结
  果提供了新的证明,并且处理了 $n = 4$ 的情况. 2010 年,Cox 和Flikkema\cite{cox2010minimal}利用数学软件
  计算了 $n \leq 42$ 的候选情况. 目前还没有利用深度学习求解该问题的论文. 

\end{frame}

\begin{frame}{球面离散划分的现有方法}

  

\end{frame}

\begin{frame}{MCTS在连续问题领域的应用案例}

  

\end{frame}

\section{数学基础与问题建模}

\begin{frame}{球面几何特征}

  

\end{frame}

\begin{frame}{优化问题形式化}

  

\end{frame}

\begin{frame}{MCTS算法改进}

  

\end{frame}



\section{实验与结果分析}



\section{结论与展望}



\section{参考文献}

\begin{frame}[allowframebreaks]{参考文献}
  
  \printbibliography[heading=none]

\end{frame}

\begin{frame}{Acknowledgement}

  \begin{center}

    \textcolor{gray}{\Huge{\centerline{\calligra{Thank you!}}}}

  \end{center}

\end{frame}

\end{document}
